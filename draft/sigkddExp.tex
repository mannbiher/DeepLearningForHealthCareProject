% It is an example file showing how to use the 'sigkddExp.cls' 
% LaTeX2e document class file for submissions to sigkdd explorations.
% It is an example which *does* use the .bib file (from which the .bbl file
% is produced).
% REMEMBER HOWEVER: After having produced the .bbl file,
% and prior to final submission,
% you need to 'insert'  your .bbl file into your source .tex file so as to provide
% ONE 'self-contained' source file.
%
% Questions regarding SIGS should be sent to
% Adrienne Griscti ---> griscti@acm.org
%
% Questions/suggestions regarding the guidelines, .tex and .cls files, etc. to
% Gerald Murray ---> murray@acm.org
%

\documentclass{sigkddExp}

\begin{document}
%
% --- Author Metadata here ---
% -- Can be completely blank or contain 'commented' information like this...
%\conferenceinfo{WOODSTOCK}{'97 El Paso, Texas USA} % If you happen to know the conference location etc.
%\CopyrightYear{2001} % Allows a non-default  copyright year  to be 'entered' - IF NEED BE.
%\crdata{0-12345-67-8/90/01}  % Allows non-default copyright data to be 'entered' - IF NEED BE.
% --- End of author Metadata ---

\title{Covid-19 X-ray Image Classification}
%\subtitle{[Extended Abstract]
% You need the command \numberofauthors to handle the "boxing"
% and alignment of the authors under the title, and to add
% a section for authors number 4 through n.
%
% Up to the first three authors are aligned under the title;
% use the \alignauthor commands below to handle those names
% and affiliations. Add names, affiliations, addresses for
% additional authors as the argument to \additionalauthors;
% these will be set for you without further effort on your
% part as the last section in the body of your article BEFORE
% References or any Appendices.

\numberofauthors{4}
%
% You can go ahead and credit authors number 4+ here;
% their names will appear in a section called
% "Additional Authors" just before the Appendices
% (if there are any) or Bibliography (if there
% aren't)

% Put no more than the first THREE authors in the \author command
%%You are free to format the authors in alternate ways if you have more 
%%than three authors.

\author{
%
% The command \alignauthor (no curly braces needed) should
% precede each author name, affiliation/snail-mail address and
% e-mail address. Additionally, tag each line of
% affiliation/address with \affaddr, and tag the
%% e-mail address with \email.
\alignauthor Maneesh Kumar Singh \\
       \affaddr{University of Illinois at Urbana-Champaign}\\
       \email{mksingh4@illinois.edu}
\alignauthor Raman Walwyn-Venugopal\\
       \affaddr{University of Illinois at Urbana-Champaign}\\
       \email{rsw2@illinois.edu}
\alignauthor Satish Reddy Asi\\
       \affaddr{University of Illinois at Urbana-Champaign}\\
       \email{sasi2@illinois.edu}
\alignauthor Srikanth Bharadwaz Samudrala\\
       \affaddr{University of Illinois at Urbana-Champaign}\\
       \email{sbs7@illinois.edu}
}

\date{30 July 1999}
\maketitle
\begin{abstract}
TODO
\end{abstract}

\section{Introduction}
As part of graduate course Deep learning for healthcare, we
have decided to reproduce and improve current research on COVID-19 classification using
X-ray.


\section{Motivation}
COVID-19 pandemic has ravaged the world on an unprecedented scale. It has caused loss
of millions of lives and long lasting damages on surviving patients. X-ray imaging
is very important part of diagnosis of Covid-19 and other pneumonia and is often the
first-line diagnosis in many cases. Using deep learning for X-ray classification is
an ongoing research area. We have taken this paper to strengthen our understanding of
deep learning models and improve the current research.


\section{Literature Survey}

\subsection{COVID-19 classification using chest CT}

X. Bai and Wang were able to create an AI system that could differentiate
COVID-19 and other pneumonia using a chest CT scan. They approached this
as a classification problem and used the EfficientNet B4 architecture which
was a CNN based network. They were able to achieve results of 96\% accuracy,
95\% sensitivity , 96\% specificity, and an area under receiver operating
characteristic curve of 0.95 and an area under the precision recall curve of
0.90. When compared with radiologists on the same test dataset, the AI system
performed better. This study concluded that the AI can support radiologists
in detection of Covid 19 in Chest CT images.

\subsection{Focal loss for dense object detection}
Lin T, Goyal P, Girshick R propose Focal loss, a modification to the standard cross
entropy criterion that focuses weights for loss on hard examples versus well
classified examples. This is accomplished by adding a factor $(1 - p_t)^\gamma$ to the
standard cross entropy criterion where setting $\gamma  > 0$ reduces the relative loss 
for well-classified examples $(p_t > .5)$. This results in achieving higher accuracy 
than using the standard cross entropy loss and surpassed speed and accuracy when 
compared with state of the art two stage detectors; Faster R CNN Variants.


\subsection{Towards contactless patient positioning}
This paper discussed about patient positioning routine that comprised a novel robust 
dynamic fusion (RDF) algorithm for accurate 3D patient body modeling. 
With ‘multi-modal’ inference capability, RDF can be trained once and used across 
different applications (without re-training). They have used multiple CNN branches 
o learn the joint feature representation and a fully-connected parameter regressor 
module to estimate the 3D mesh parameters.


\subsection{Deep learning COVID-19 features on CXR using limited training data sets}
The authors of this paper, proposed a patch-based convolutional neural network approach
with a relatively small number of trainable parameters for Covid-19 diagnosis.
The architecture contains first pre-processed data that are fed into a segmentation
network [FC-DenseNet] to extract lung areas. From this segmented lung area, classification
network is used to classify the corresponding diseases using a patch-by-patch training
and inferences [ResNet-18 (pre-trained) and many ResNet-18 models are used for K patches],
final decision is made based on the majority voting from previous layers. A Grad-CAM
saliency map is calculated to provide an interpretable result. This method has an accuracy
of 91.9\%, compared to that of 92.4\% for COVID-Net.

\subsection{COVID19-Net Deep Convolutional Neural Network}

This is the first open source network design for COVID-19 detection from CXR images,
our final research paper also considers this as its baseline for experiments.
This paper considered COVIDx dataset which contains 13,975 CXR images for training and
experiments. COVID-Net architecture makes heavy use of a lightweight residual
‘projection expansion projection extension’ (PPEX) design pattern that contains multiple
levels of convolution layers with fully connected layers and a softmax at the end.
COVID-Net achieved higher test accuracy than other architectures such as VGG-19 and ResNet-50.

\subsection{Noise-robust segmentation of COVID-19 from CT images}

This is a CNN model developed to be effective with detection of COVID-19 lesions from 
CT images that have a lot of noise. This paper discusses how Wang et al developed a
novel noise-robust learning framework based on self-ensembling of CNNs.  To better
deal with the complex lesions, a novel COVID-19 Pneumonia Lesion segmentation network
(COPLE-Net) was proposed that uses a combination of max-pooling and average pooling to
reduce information loss during downsampling, and employs bridge layers to alleviate the
semantic gap between features in the encoder and decoder. Experimental results with CT
images of 558 COVID-19 patients showed the effectiveness of the noise-robust Dice loss
function, COPLE-Net and adaptive self-ensembling in learning from noisy labels for COVID-19
pneumonia lesion segmentation. To make the training process robust against noisy labels,
a novel noise-robust Dice loss function was proposed and integrated into a self-ensembling
framework, where an adaptive teacher and an adaptive student are introduced to further improve
the performance in dealing with noisy labels. The experiments used 2D CNNs for slice-by-slice
segmentation, and implemented COPLE-Net, 1 LNR-Dice and the adaptive self-ensembling framework
in Pytorch with the PyMIC 3 library on a Ubuntu desktop with an NVIDIA GTX 1080 Ti GPU. 

The proposed COPLE-Net was compared with four state-of-the-art networks for semantic or medical
image segmentation

\begin{enumerate}
       \item 3D nnU-Net that is extended from 3D U-Net
       \item Attention U-Net 3) ScSE U-Net
       \item ESPNetv2 and proven to be most effective with noisy images.
\end{enumerate}

In addition, COPLE-Net was compared with three variants: COPLE-Net (-A), COPLE-Net (-D)
and COPLE-Net (-B) 

%
%You can also use a citation as a noun in a sentence, as
% is done here, and in the \citeN{herlihy:methodology} article;
% use \texttt{{\char'134}citeN} in this case.  You can
% even say, ``As was shown in \citeyearNP{bowman:reasoning}. . .''
% or ``. . . which agrees with \citeANP{braams:babel}...'',
% where the text shows only the year or only the author
% component of the citation; use \texttt{{\char'134}citeyearNP}
% or \texttt{{\char'134}citeANP}, respectively,
% for these.  Most of the various citation commands may
% reference more than one work \cite{herlihy:methodology,bowman:reasoning}.
% A complete list of all citation commands available is
% given in the \textit{Author's Guide}.

\subsection{Data}

The following datasets that we are considering for this project:
\begin{enumerate}
       
\item Covid Chest X-ray (CCX) dataset: This dataset contains COVID-19 pneumonia images 
as well few X-ray images from other classes. The dataset can be obtained from 
github at this link   https://github.com/ieee8023/covid-chestxray-dataset
\item Kaggle Chest X-ray (KCX) dataset: This dataset contains normal, bacterial pneumonia, 
and nov-COVID-19 viral pneumonia. The dataset can be obtained from Kaggle at this 
link https://www.kaggle.com/paultimothymooney/chest-xray-pneumonia
\end{enumerate}


In the FLANNEL paper, 5508 chest x-ray images across 2874 independent patient cases. Both dataset contains anteroposterior (AP) and posteroanterior (PA) view.

 As done in the research paper, we will include  both AP and PA views. Due to AP and PA views being different types of X-ray images, horizontal flips and random noise will be used to convert PA into AP view. 

\subsection{Tables}
Because tables cannot be split across pages, the best
placement for them is typically the top of the page
nearest their initial cite.  To
ensure this proper ``floating'' placement of tables, use the
environment \textbf{table} to enclose the table's contents and
the table caption.  The contents of the table itself must go
in the \textbf{tabular} environment, to
be aligned properly in rows and columns, with the desired
horizontal and vertical rules.  Again, detailed instructions
on \textbf{tabular} material
is found in the \textit{\LaTeX\ User's Guide}.

Immediately following this sentence is the point at which
Table 1 is included in the input file; compare the
placement of the table here with the table in the printed
dvi output of this document.

\begin{table}
\centering
\caption{Frequency of Special Characters}
\begin{tabular}{|c|c|l|} \hline
Non-English or Math&Frequency&Comments\\ \hline
\O & 1 in 1,000& For Swedish names\\ \hline
$\pi$ & 1 in 5& Common in math\\ \hline
\$ & 4 in 5 & Used in business\\ \hline
$\Psi^2_1$ & 1 in 40,000& Unexplained usage\\
\hline\end{tabular}
\end{table}

To set a wider table, which takes up the whole width of
the page's live area, use the environment
\textbf{table*} to enclose the table's contents and
the table caption.  As with a single-column table, this wide
table will "float" to a location deemed more desirable.
Immediately following this sentence is the point at which
Table 2 is included in the input file; again, it is
instructive to compare the placement of the
table here with the table in the printed dvi
output of this document.


\begin{table*}
\centering
\caption{Some Typical Commands}
\begin{tabular}{|c|c|l|} \hline
Command&A Number&Comments\\ \hline
\texttt{{\char'134}alignauthor} & 100& Author alignment\\ \hline
\texttt{{\char'134}numberofauthors}& 200& Author enumeration\\ \hline
\texttt{{\char'134}table}& 300 & For tables\\ \hline
\texttt{{\char'134}table*}& 400& For wider tables\\ \hline\end{tabular}
\end{table*}
% end the environment with {table*}, NOTE not {table}!

\subsection{Theorem-like Constructs}
Other common constructs that may occur in your article are
the forms for logical constructs like theorems, axioms,
corollaries and proofs.  There are
two forms, one produced by the
command \texttt{{\char'134}newtheorem} and the
other by the command \texttt{{\char'134}newdef}; perhaps
the clearest and easiest way to distinguish them is
to compare the two in the output of this sample document:

This uses the \textbf{theorem} environment, created by
the \texttt{{\char'134}newtheorem} command:
\newtheorem{theorem}{Theorem}
\begin{theorem}
Let $f$ be continuous on $[a,b]$.  If $G$ is
an antiderivative for $f$ on $[a,b]$, then
\begin{displaymath}\int^b_af(t)dt = G(b) - G(a).\end{displaymath}
\end{theorem}

The other uses the \textbf{definition} environment, created
by the \texttt{{\char'134}newdef} command:
\newdef{definition}{Definition}
\begin{definition}
If $z$ is irrational, then by $e^z$ we mean the
unique number which has
logarithm $z$: \begin{displaymath}{\log_e^z = z}\end{displaymath}
\end{definition}

Two lists of constructs that use one of these
forms is given in the
\textit{Author's  Guidelines}.
 
There is one other similar construct environment, which is
already set up
for you; i.e. you must \textit{not} use
a \texttt{{\char'134}newdef} command to
create it: the \textbf{proof} environment.  Here
is a example of its use:
\begin{proof}
Suppose on the contrary there exists a real number $L$ such that
\begin{displaymath}
\lim_{x\rightarrow\infty} \frac{f(x)}{g(x)} = L.
\end{displaymath}
Then
\begin{displaymath}
l=\lim_{x\rightarrow c} f(x)
= \lim_{x\rightarrow c}
\left[ g{x} \cdot \frac{f(x)}{g(x)} \right ]
= \lim_{x\rightarrow c} g(x) \cdot \lim_{x\rightarrow c}
\frac{f(x)}{g(x)} = 0\cdot L = 0,
\end{displaymath}
which contradicts our assumption that $l\neq 0$.
\end{proof}

Complete rules about using these environments and using the
two different creation commands are in the
\textit{Author's Guide}; please consult it for more
detailed instructions.  If you need to use another construct,
not listed therein, which you want to have the same
formatting as the Theorem
or the Definition\cite{salas:calculus} shown above,
use the \texttt{{\char'134}newtheorem} or the
\texttt{{\char'134}newdef} command,
respectively, to create it.

\section{Experimental Setup}
AWS (custom cluster/SageMaker)
pandas 
Pytorch


\section{Conclusions}
This paragraph will end the body of this sample document.
Remember that you might still have Acknowledgements or
Appendices; brief samples of these
follow.  There is still the Bibliography to deal with; and
we will make a disclaimer about that here: with the exception
of the reference to the \LaTeX\ book, the citations in
this paper are to articles which have nothing to
do with the present subject and are used as
examples only.
%\end{document}  % This is where a 'short' article might terminate

%ACKNOWLEDGEMENTS are optional
\section{Acknowledgements}
This section is optional; it is a location for you
to acknowledge grants, funding, editing assistance and
what have you.  In the present case, for example, the
authors would like to thank Gerald Murray of ACM for
his help in codifying this \textit{Author's Guide}
and the \textbf{.cls} and \textbf{.tex} files that it describes.

%
% The following two commands are all you need in the
% initial runs of your .tex file to
% produce the bibliography for the citations in your paper.
\bibliographystyle{abbrv}
\bibliography{sigproc}  % sigproc.bib is the name of the Bibliography in this case
% You must have a proper ".bib" file
%  and remember to run:
% latex bibtex latex latex
% to resolve all references
%
% ACM needs 'a single self-contained file'!
%
%APPENDICES are optional
% SIGKDD: balancing columns messes up the footers: Sunita Sarawagi, Jan 2000.
% \balancecolumns
\appendix
%Appendix A
\section{Headings in Appendices}
The rules about hierarchical headings discussed above for
the body of the article are different in the appendices.
In the \textbf{appendix} environment, the command
\textbf{section} is used to
indicate the start of each Appendix, with alphabetic order
designation (i.e. the first is A, the second B, etc.) and
a title (if you include one).  So, if you need
hierarchical structure
\textit{within} an Appendix, start with \textbf{subsection} as the
highest level. Here is an outline of the body of this
document in Appendix-appropriate form:
\subsection{Introduction}
\subsection{The Body of the Paper}
\subsubsection{Type Changes and Special Characters}
\subsubsection{Math Equations}
\paragraph{Inline (In-text) Equations}
\paragraph{Display Equations}
\subsubsection{Citations}
\subsubsection{Tables}
\subsubsection{Figures}
\subsubsection{Theorem-like Constructs}
\subsubsection*{A Caveat for the \TeX\ Expert}
\subsection{Conclusions}
\subsection{Acknowledgements}
\subsection{Additional Authors}
This section is inserted by \LaTeX; you do not insert it.
You just add the names and information in the
\texttt{{\char'134}additionalauthors} command at the start
of the document.
\subsection{References}
Generated by bibtex from your ~.bib file.  Run latex,
then bibtex, then latex twice (to resolve references)
to create the ~.bbl file.  Insert that ~.bbl file into
the .tex source file and comment out
the command \texttt{{\char'134}thebibliography}.
% This next section command marks the start of
% Appendix B, and does not continue the present hierarchy
\section{More Help for the Hardy}
The acmproc-sp document class file itself is chock-full of succinct
and helpful comments.  If you consider yourself a moderately
experienced to expert user of \LaTeX, you may find reading
it useful but please remember not to change it.

% That's all folks!
\end{document}
